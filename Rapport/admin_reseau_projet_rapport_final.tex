\documentclass[a4paper,12pt]{article}


\usepackage[utf8]{inputenc}
\usepackage[T1]{fontenc}
\usepackage[french]{babel}
\usepackage{lmodern}

\usepackage{vmargin}
\usepackage{graphicx}
\usepackage{tabularx}
\usepackage{amsmath}
\usepackage{amssymb}

\let\oldtheta\theta
\renewcommand{\theta}{\ensuremath{\oldtheta}}

%\addtolength{\textwidth}{1cm}
%\usepackage{geometry}
\setlength{\parindent}{0mm}

\renewcommand{\arraystretch}{2}

\begin{document}

% Page de titre
\titlepage{
	\today \\[2cm]
	\begin{flushright}\sf\Huge
	{\bfseries Administration système et réseaux II} \\[3mm]
	{\huge Rapport final}
	\end{flushright}
	\ \\[2cm]
	\begin{center}
	
\end{center}
	\vspace{4cm}
	Samuel Petre, Quentin Puttemans, Arnaud Renard - Groupe 14 - 2TL2\\
}

\clearpage

%\vspace{5cm}
\section{Cahier des charges}
	\paragraph{} L'entreprise Woodytoys nous demande d'implémenter une architecture permettant la mise en place de trois services web. Le nom de domaine de base sera wt14.ephec-ti.be. Nous aurons donc trois noms de domaine à gérer : \begin{itemize}
	\item www.wt14.ephec-ti.be
	\item b2b.wt14.ephec-ti.be
	\item intranet.wt14.ephec-ti.be
	\end{itemize}
	\paragraph{} www.wt14.ephec-ti.be sera un site statique.
	\paragraph{} b2b.wt14.ephec-ti.be sera un site dynamique comportant une base de données permettant aux revendeurs de passer des commandes.
	\paragraph{} intranet.wt14.ephec-ti.be sera un site permettant aux employés d'avoir accès à un l'intranet. Ce site sera donc impossible à atteindre depuis l'Internet.
	
	\paragraph{} L’entreprise souhaite aussi disposer d’une adresse mail liée à son domaine. C’est
à dire, de type nom.prenom@wt14.ephec-ti.be. De plus, il devra être possible de
créer des adresses mail génériques, telles que contact@wt14.ephec-ti.be qui seront
redirigées vers la personne responsable du service.
	\paragraph{} Les employés doivent pouvoir consulter et envoyer des e-mails aussi bien à l’intérieur
de l’entreprise qu’à l’extérieur.
	\paragraph{} L’entreprise souhaite aussi que l’ensemble de ses employés puisse communiquer
entre eux via des postes téléphoniques IP. Certains employés devront aussi être
capables de communiquer avec le monde extérieur. Il existe différents types d’employés
: \begin{itemize}
	\item Les ouvriers : Ils disposent d’un poste de téléphonie IP dans leur atelier pour
joindre les autres départements internes.
	\item La secrétaire : Elle dispose d’un PC sur lequel se trouve un softphone, lui permettant
de contacter n’importe qui.
	\item Le service comptable : réparti dans deux bureaux, il dispose d’un numéro unique
permettant de joindre le premier comptable disponible, ainsi que d’un numéro
spécifique par bureau. Les comptables peuvent joindre l’extérieur et tout le monde
en interne à l’exception du directeur.
	\item Les commerciaux : réunis dans un même bureau, ils peuvent joindre l’extérieur
et tout le monde en interne à l’exception du directeur. Ils disposent également de
smartphones avec lesquels ils peuvent téléphoner en déplacement.
	\item La direction : Un numéro qui peut joindre tous les autres postes internes ainsi
que l’extérieur. Ce numéro ne peut pas être joint directement, les appels devant
transiter préalablement par la secrétaire.
\end{itemize}
	\paragraph{} Les employés doivent chacun disposer d’une boite vocale.
	
	\paragraph{} L’entreprise Woodytoys allant bientôt fusionner avec une entreprise possédant
une structure interne similaire, il nous est demandé que les employés puissent aussi
contacter les membres de cette autre entreprise.
	\paragraph{}L’entreprise souhaiterait aussi, mais optionnellement, un service de partage de
fichiers afin de pouvoir centraliser les documents utilisés par les employés.
Chaque employé doit disposer de son répertoire personnel, de même que le directeur
et la secrétaire.
Chaque groupe d’employés doit disposer d’un répertoire commun.
Les employés doivent pouvoir accéder aux fichiers partagés via l’explorateur natif
du système
Le système doit pouvoir être sauvegardé en copie (backup) facilement.
Les employés doivent avoir la possibilité d’accéder à leurs fichiers à l’extérieur de
l’entreprise.
	\paragraph{}Nous devons donc implémenter et configurer adéquatement un serveur web, un
résolveur DNS, une base de données, trois pages web, un serveur SMTP, un serveur
IMAP/POP3, un service VoIP, ainsi qu’un serveur de partage de fichiers (optionnel).	
	
	
\section{Proposition de solutions techniques}
	\paragraph{} Afin de pouvoir accéder aux 3 pages web demandées, nous avions premièrement
besoin d’un serveur web. Nous avons choisi d’utiliser le serveur web Apache, car c’est
le web serveur le plus répandu sur les machines Linux, et le deuxième plus répandu,
tous systèmes d’exploitation confondus. Le serveur web Apache ayant été le serveur
le plus populaire depuis 1996, celui-ci dispose d’une documentation très complète.
Sa popularité n’est bien sûr pas le seul élément ayant contribué à notre choix, car,
en effet, Apache perd en popularité, contrairement à son concurrent, Nginx. Ce qui
nous a fait décider d’utiliser Apache est tout simplement son support interne natif
pour les contenus dynamiques, contrairement à Nginx qui nécessite une intervention
extérieure. Ce support natif rend la configuration plus simple à déployer.
	\paragraph{} Pour le site d’e-commerce, nous avions aussi besoin d’une base de données, afin de
pouvoir enregistrer l’ensemble des produits. Nous avons renoncé notre idée de base
qui était d’installer la base de données sur un serveur distant, afin de ne pas être
dépendant au niveau sécurité, mais aussi accessibilité du serveur distant. En effet, si
le serveur distant venait à fermer, il n’y aurait plus aucun accès à la base de donnée,
ni même de sauvegarde. En décidant de mettre notre base de données sur un serveur
local, nous avons un contrôle intégral dessus. Nous avons choisi d’utiliser un serveur
MySQL, celui-ci étant le plus efficace pour la tâche à accomplir. De plus, cela nous
permettait d’utiliser le web stack LAMP (Linux, Apache, MySQl, PHP), qui est une
combinaison de programmes complètement gratuits et open-source, ce qui est aussi
signe de plus de sécurité : les failles étant découvertes et réparées plus souvent.
\\
\\
	\paragraph{} Troisièmement, pour accéder aux 3 pages web demandées, nous avions aussi besoin
d’un résolveur DNS. Cela permet de pouvoir visiter les pages web via leur nom et
pas l’adresse IP du serveur sur lequel elles se trouvent. Nous utilisons ici le BIND,
qui est le serveur DNS le plus utilisé sur internet (79% des serveurs en 2008).
	\paragraph{} En ce qui concerne l’infrastructure mail, nous avions besoin d’un serveur SMTP
qui se charge d’envoyer des mails, et un serveur POP3/IMAP qui se charge de récupérer
les mails. Comme serveur SMTP, nous avons choisi d’utiliser Postfix pour
sa vitesse défiant toute concurrence, sa robustesse et sa sécurité. Nous avons privilégié
Postfix par rapport à d’autres serveurs SMTP (Exim, par exemple), car il a la
particularité d’être très modulaire, ce qui permet de n’utiliser et configurer que ce
qu’on a besoin.
	\paragraph{} Nous avions ensuite le choix entre utiliser POP3 ou IMAP pour recevoir et lire les
mails, ou les deux. Nous avons choisi de n’utiliser que IMAP, celui-ci permettant la
synchronisation de plusieurs appareils. En effet, POP3 ne permet de télécharger les
mails que sur un seul appareil. Nous utilisons Dovecot, car il va de pair et fonctionne
très bien avec Postfix.
	\paragraph{} Nous voulions aussi installer Squirrelmail, qui permet de consulter ses mails depuis
un navigateur internet (contrairement à un logiciel si Squirrelmail n’est pas installé),
mais par manque de temps, nous avons abandonné l’idée.
	\paragraph{} Nous devions ensuite déployer un moyen de communication par voix par IP. Pour
ce faire, nous avions besoin d’un logiciel permettant l’appel gratuit entre employés
à l’intérieur d’une entreprise. Nous nous sommes vite penché le logiciel Asterisk.
Asterisk est devenu un standard dans les systèmes modernes de voix par IP. C’est
un logiciel open-source, puissant et flexible. Cela permet le développement de l’entreprise
Woodytoys sans devoir trop changer le système de voix par IP.
	\paragraph{} Une autre alternative à Asterisk était FreeSwitch, mais Freeswitch demande une
machine plus performante pour bien fonctionner. Notre machine de développement
n’étant pas suffisamment puissante, nous avons opté pour Asterisk, qui lui, peut
tourner sur des machines très basiques.
	\paragraph{} La solution actuelle peut gérer 99 employés de chaque branche (y compris secrétaires
et directeurs) pour la structure locale. L’entreprise avec laquelle la fusion
s’opère peut aussi gérer 99 employés dans chaque branche. Le 100e numéro est destiné
aux différentes boites vocales.

	\paragraph{} Pour le système de partage de fichiers, nous avons très vite opté pour une solution rendant facile l'accès aux fichiers. Nous avons donc choisi Owncloud, celui-ci permettant facilement l'accès via un navigateur web. De plus, Owncloud est un outil open source et un des outil les plus populaire en ce qui concerne le partage de fichiers via un serveur privé.

\section{Rapport sur ce qui a été déployé}
	\paragraph{} En ce qui concerne la partie web, les trois pages sont accessibles respectivement
sous les URL 
\begin{itemize}
	\item www.wt14.ephec-ti.be
	\item b2b.wt14.ephec-ti.be
	\item intranet.wt14.ephec-ti.be
\end{itemize}
	\paragraph{} La page www est une page statique, la page b2b est une page dynamique, communiquant
avec une base de données. Ces deux pages sont accessibles à tout le monde.
La page intranet, elle, n’est disponible qu’à l’intérieur de l’entreprise. Essayer d’y
accéder en dehors affichera une page d’erreur.
	\paragraph{} La base de données est déployée en local et est fonctionnelle. Elle est utilisée via
le site dynamique b2b.wt14.ephec-ti.be.
	\paragraph{} Pour ce qui est des mails, il est possible de les consulter uniquement à l’intérieur
de l’entreprise. Il est possible de recevoir des mails de tout le monde (interne et
externe), mais il n’est possible d’en envoyer qu’à partir de l’intérieur de l’entreprise
vers un collègue à l’intérieur de l’entreprise.
	\paragraph{} Les adresses mail génériques sont fonctionnelles et redirigent pour le moment vers
un compte par défaut.
	\paragraph{} Pour ajouter, supprimer, ou modifier une adresse mail ou une adresse mail générique,
il faut se référer au wiki du Github (mode d’emploi).
	\paragraph{} Pour le système de communication par voix par IP, l’ensemble des employés
peuvent communiquer entre eux. Seule la secrétaire peut contacter directement le
directeur. De plus, les comptables possèdent un numéro permettant d’appeler le
premier comptable disponible.
	\paragraph{} Chaque membre de l’entreprise possède une boite vocale. Nous avons choisi de
donner un numéro de boite vocale différent par section (secrétaire, comptable, ouvriers,
etc.) afin d’assurer une meilleure organisation.
	\paragraph{} La fusion avec une autre entreprise a été un succès. Il est donc possible pour l’ensemble
des employés de chaque entreprise de communiquer avec leurs collègues se
trouvant dans l’autre entreprise et étant connecté au réseau de leur entreprise.

	\paragraph{} Le Owncloud est entièrement déployé. Un groupe a été créé pour chaque branche de l'entreprise, et des employés de base y ont été ajouté. Chaque membre dispose d'un répertoire personnel, de plus qu'un répertoire partagé avec l'ensemble des membres de sa branche. Le directeur dirige toutes les branches, et a donc la permission d'ajouter, supprimer et modifier le compte d'un employé. 
	
\section{Explication de la procédure de validation}
	\paragraph{} Pour vérifier le fonctionnement du site statique, nous nous sommes tout simplement
rendus sur http ://www.wt14.ephec-ti.be/ via une machine externe afin d’être
certains que le site soit accessible depuis internet.
	\paragraph{} Nous avons fait de même pour le site dynamique, mais cette fois-ci avec le lien
http ://b2b.wt14.ephec-ti.be/. 
	\paragraph{} Pour l’intranet, nous avons vérifié qu’accéder au site depuis une machine externe à
l’entreprise (ici, toutes autres machines que nos 3 vps) nous renvoie bien sur la page
d’erreur et non la page de l’intranet. Ensuite, une fois cela fait, nous avons accédé
au site depuis une machine de l’entreprise, et cette fois-ci, la page de l’intranet était
bien disponible.
	\paragraph{} La procédure de vérification du mail fut, quant à elle, plus longue. Premièrement,
nous avons vérifié que nous pouvions envoyer et recevoir des mails en local. Pour ce
faire, nous avons créé deux nouvelles adresses mail et avec l’une d’entre elles, nous
avons envoyé un message à l’autre. L’autre a bien reçu et peut lire le message avec
succès. Ensuite, nous avons envoyé un mail à une adresse interne depuis une adresse
Gmail. Le mail a été envoyé et reçu avec succès.
	\paragraph{} Nous avons ensuite essayé d’envoyer un mail depuis une adresse locale vers une
adresse Gmail, mais les messages n’arrivent jamais. Pire encore, ils sont détectés
comme du spam. Nous pensons que cela vient du fait que nous n’avons pas implémenté
de certificats SSL. Nous pouvons nous connecter sur notre compte mail
depuis l’extérieur via un logiciel comme Thunderbird, mais il n’est pas possible de
récupérer nos e-mails. Encore une fois, nous pensons à un problème lié au manque
de certificats SSL.
	\paragraph{} Pour le système de voix par IP Asterisk, nous avons tout d’abord testé que l’on
peut se connecter au serveur soit en utilisant l’adresse IP, soit en utilisant le nom de
domaine (voip.wt14.ephec-ti.be). Une fois que cela a fonctionné, nous avons testé un
par un chaque service demandé dans le cahier des charges (seule la secrétaire peut
appeler le directeur, etc.). L’ensemble des fonctionnalités fonctionne, à l’exception
que nous n’avons pas testé si l’on pouvait joindre l’extérieur, et si l’extérieur peut
joindre la secrétaire.
	\paragraph{} Pour ce qui est de la fusion d’entreprise, nous avons repris les mêmes démarches
que plus haut, à l’exception qu’une personne était connectée à une entreprise, et
l’autre était connectée à l’autre entreprise.

	\paragraph{} Pour valider le Owncloud, nous nous sommes tout simplement rendu sur cloud.wt14.ephec-ti.be:8080 et nous avons testé le dossiers partagés ainsi que les dossiers personnels de plusieurs membres.

\section{Sources}
	\begin{itemize}
		\item https ://news.netcraft.com/archives/2016/02/22/february-2016-web-server-survey.html
(Répartition des différents serveurs web sur le marché)
		\item https ://fr.wikipedia.org/wiki/BIND (Pourcentage d’utilisation de BIND)
		\item https ://www.whichvoip.com/freeswitch-vs-asterisk.htm (Asterisk-FreeSwitch : Système
requis)
	\end{itemize}

\end{document}