\documentclass[a4paper,12pt]{article}


\usepackage[utf8]{inputenc}
\usepackage[T1]{fontenc}
\usepackage[french]{babel}
\usepackage{lmodern}

\usepackage{vmargin}
\usepackage{graphicx}
\usepackage{tabularx}
\usepackage{amsmath}
\usepackage{amssymb}

\let\oldtheta\theta
\renewcommand{\theta}{\ensuremath{\oldtheta}}

%\addtolength{\textwidth}{1cm}
%\usepackage{geometry}
\setlength{\parindent}{0mm}

\renewcommand{\arraystretch}{2}

\begin{document}

% Page de titre
\titlepage{
	\today \\[2cm]
	\begin{flushright}\sf\Huge
	{\bfseries Administration système et réseaux II} \\[3mm]
	{\bfseries Module 3 - Mise en place d’une
infrastructure VoIP
} \\[2mm]
	{\huge Rapport module 3}
	\end{flushright}
	\ \\[2cm]
	\begin{center}
	
\end{center}
	\vspace{4cm}
	Samuel Petre, Quentin Puttemans, Arnaud Renard - Groupe 14 - 2TL2\\
}

\clearpage

%\vspace{5cm}
\section{Cahier des charges}
	\paragraph{} L’entreprise souhaite adopter une téléphonie basée sur IP. Cela signifie
deux choses :
 \begin{itemize}
	\item L’entreprise doit être accessible en VoIP depuis Internet, afin que des
clients puissent la contacter. Les appels doivent aboutir sur le poste
de la secrétaire.
	\item Les employés disposent de softphones sur leurs ordinateurs, à l’exception
des ouvriers qui ont un poste de téléphonie IP à disposition.

	\item Les employés de l’entreprise doivent pouvoir communiquer 		entre eux,
à l’intérieur de l’entreprise, mais également depuis l’extérieur dans
le cas des commerciaux qui sont souvent en déplacement. Les types
d’employés sont les suivants :
	\begin{itemize}
		\item Les ouvriers : Ils disposent d’un poste de téléphonie IP dans leur
atelier pour joindre les autres départements internes.

		\item La secrétaire : Elle dispose d’un PC sur lequel se trouve un softphone,
lui permettant de contacter n’importe qui.
		\item Le service comptable : Réparti dans deux bureaux, il dispose d’un numéro unique permettant de joindre le premier comptable disponible,
ainsi que d’un numéro spécifique par bureau. Les comptables
peuvent joindre l'extérieur et tout le monde en interne à
l’exception du directeur.
		\item Les commerciaux : Réunis dans un même bureau, ils peuvent
joindre l’extérieur et tout le monde en interne à l'exception du
directeur. Ils disposent également de smartphones avec lesquels ils
peuvent téléphoner en déplacement.
		\item La direction : Un numéro qui peut joindre tous les autres postes
internes ainsi que l’extérieur. Ce numéro ne peut pas être joint
directement, les appels devant transiter préalablement par la secrétaire.
	\end{itemize}
	\item Les employés doivent pouvoir disposer d’une boîte vocale.
	\end{itemize}
		
	
\section{Proposition de solutions techniques}
	\paragraph{} Nous avons directement pensé à utiliser Asterisk car c'est un système comportant énormément de ressources et capable de faire énormément de choses. C'est à dire, que ce système sera capable de supporter la majorité des demandes du futur.
	\paragraph{} La configuration actuelle est capable de gérer 1000 employés par secteur (ouvriers, comptables, secrétaires, etc). 
	\paragraph{} Pour ajouter un nouveau numéro pour, par exemple, un nouvel employé, il faut aller dans le fichier users.conf et en dessous du bon secteur (par exemple [ouvriers](commun)), créer un nouveau numéro de la façon suivante : \newline \newline
	[2003](ouvriers)\newline
	username=ouvriers3\newline
	secret=passwd\newline
	
	2003 sera le numero de telephone de l'employé, (ouvriers) est le type de l'employé, username est son nom et secret est son mot de passe.
	\paragraph{} Si un comptable est rajouté, il faudra se rendre dans le fichier extensions.conf et à la ligne \newline \newline exten => 3999,1,Dial(SIP/3001\&SIP/3002,20) \newline \newline ajouter avant le ",20)" "SIP/3XXX" avec 3XXX le numero du nouveau comptable. par exemple un nouveau comptable avec le numero 3003, nous aurions : \newline \newline exten => 3999,1,Dial(SIP/3001\&SIP/3002\&SIP/3003,20) \newline \newline C'ets cette ligne qui permet que tous les comptables soient appelés lorsque le numero 3999 est appelé.
		
\section{Rapport sur ce qui a été déployé}
	\paragraph{} L'ensemble des fonctionnalités de base ont été implémentées. Il est possible de passer des appels en respectant le cahier des charges donné. 
	\paragraph{} Il ne reste qu'une petite chose à améliorer : la redirection d'appel pour que la secrétaire puisse rediriger les appels vers la direction.
	
\section{Explication de la procédure de validation}
	\paragraph{} Pour vérifier le fonctionnement de la voix par IP, nous avons utilisé le logiciel X-LITE sur un PC fixe, et l'application Zoiper sur Ipod. Nous nous sommes ensuite chacun connecté sur des comptes différents et nous avons réussi à nous joindre avec succès. 

\end{document}